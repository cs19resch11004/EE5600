\documentclass[journal,12pt,twocolumn]{IEEEtran}

\usepackage{graphicx}
\graphicspath{ {./5609/Assign4/} }

\makeatletter
\newcommand\xleftrightarrow[2][]{%
  \ext@arrow 9999{\longleftrightarrowfill@}{#1}{#2}}
\newcommand\longleftrightarrowfill@{%
  \arrowfill@\leftarrow\relbar\rightarrow}
\makeatother


\newcommand\leftarrowS{\leftarrow\joinrel\smalldollar}
\newcommand\rightarrowS{\smalldollar\joinrel\rightarrow}


\usepackage{amsmath}
\newcommand{\myvec}[1]{\ensuremath{\begin{pmatrix}#1\end{pmatrix}}}
\title{Assignment 5}
\author{PARIMISETTY HARINADHA (CS19RESCH11004)}
\begin{document}
\maketitle
\newpage
\begin{abstract}
This assignment deals with QR decomposition .
\end{abstract}
Download all python codes from 
\framebox[1\width]{ https://github.com/cs19resch11004/hari }
Download all Latex-tikz codes from 
\framebox[1.1\width]{ https://github.com/cs19resch11004/hari} 
\section{Problem}
Perform QR decomposition on matrix V
\begin{align}
    V=\myvec{ 1 & \frac{1}{2} \\ \frac{1}{2} & 1 }
\end{align}
\section{Solution}
The columns of matrix V can be represented in $\vec{\alpha}$ and $\vec{\beta}$ as
\begin{align}
    \implies\vec{\alpha}=\myvec{1 \\ \frac{1}{2}}\\
    \vec{\beta}=\myvec{\frac{1}{2}\\ 1}
\end{align}

For QR decomposition, matrix V can be expressed as
\begin{align}
    V=QR \label{eq:4}
\end{align}

where, Q and R are expressed as
\begin{align}
    Q=\myvec{\vec{u_1} & \vec{u_2}} \label{eq:5}\\
    R=\myvec{k_1 & r_1 \\ 0 & k_2} \label{eq:6}
\end{align}
Note that R is an upper triangular matrix.\\

Now,we calculate
\begin{align}
    k_1=\| \mathbf{\vec{\alpha}} \|=\frac{\sqrt{5}}{2}\\
    \vec{u_1}=\frac{\vec{\alpha}}{k_1}=\frac{2}{\sqrt{5}}\myvec{1 \\ \frac{1}{2}}=\myvec{\frac{2}{\sqrt{5}} \\ \frac{1}{\sqrt{5}}}\\
r_1=\frac{\vec{u_1}^T\vec{\beta}}{\| \mathbf{\vec{u_1}} \|^2}=\myvec{\frac{2}{\sqrt{5}} & \frac{1}{\sqrt{5}}}\myvec{\frac{1}{2} \\ 1}=\frac{2}{\sqrt{5}}\\
\vec{u_2}=\frac{\vec{\beta}-r_1\vec{u_1}}{\| \vec{\beta}-r_1\vec{u_1} \|}\label{eq:10}
\end{align}

Consider
\begin{align}
    \vec{\beta}-r_1\vec{u_1}&=\myvec{\frac{1}{2} \\\\ 1}-\frac{2}{\sqrt{5}}\myvec{\frac{2}{\sqrt{5}} \\\\ \frac{1}{\sqrt{5}}}\\
\beta-r_1\vec{u_1}&=\myvec{\frac{-3}{10} \\\\ \frac{3}{5}} \label{eq:12}\\
\| \beta-r_1\vec{u_1} \|&=\frac{{2\sqrt{5}}}{3} \label{eq:13}
\end{align}

Substitute equation \eqref{eq:12}, equation \eqref{eq:13} in an equation \eqref{eq:10}, we get

\begin{align}
    \vec{u_2}&=\myvec{\frac{-1}{\sqrt{5}} \\\\ \frac{2}{\sqrt{5}}}\\
    k_2&=\vec{u_2}^T\beta=\myvec{\frac{-1}{\sqrt{5}} & \frac{2}{\sqrt{5}}}\myvec{\frac{1}{2} \\\\ 1}\\
    \implies k_2&=\frac{3}{2\sqrt{5}}
\end{align}

Therefore, from \eqref{eq:5} and \eqref{eq:6}
\begin{align}
    Q=\myvec{\frac{2}{\sqrt{5}} & \frac{-1}{\sqrt{5}} \\ \frac{1}{\sqrt{5}} & \frac{2}{\sqrt{5}}}\\
    R=\myvec{\frac{\sqrt{5}}{2} & \frac{2}{\sqrt{5}} \\ 0 & \frac{3}{2\sqrt{5}}}
\end{align}

Note that,
\begin{align}
    Q^TQ&=\myvec{\frac{2}{\sqrt{5}} & \frac{1}{\sqrt{5}} \\ \frac{-1}{\sqrt{5}} & \frac{2}{\sqrt{5}}}\myvec{\frac{2}{\sqrt{5}} & \frac{-1}{\sqrt{5}} \\ \frac{1}{\sqrt{5}} & \frac{2}{\sqrt{5}}}=\myvec{1 & 0\\ 0 & 1}=I\\
Q^TQ&=\myvec{1 & 0\\ 0 & 1}=I
\end{align}

Now matrix V can be written as \eqref{eq:4}
\begin{align}
    \myvec{1 & \frac{1}{2} \\ \frac{1}{2} & 1}=\myvec{\frac{2}{\sqrt{5}} & \frac{-1}{\sqrt{5}} \\ \frac{1}{\sqrt{5}} & \frac{2}{\sqrt{5}}}\myvec{\frac{\sqrt{5}}{2} & \frac{2}{\sqrt{5}} \\ 0 & \frac{3}{2\sqrt{5}}}
\end{align}
\end{document}
